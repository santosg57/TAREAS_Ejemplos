\documentclass[10pt]{article}

\usepackage{vmargin}

\setpapersize{A4}
\setmargins{2.5cm}       % margen izquierdo
{1.5cm}                        % margen superior
{16.5cm}                      % anchura del texto
{23.42cm}                    % altura del texto
{10pt}                           % altura de los encabezados
{1cm}                           % espacio entre el texto y los encabezados
{0pt}                             % altura del pie de p�gina
{2cm}                           % espacio entre el texto y el pie de p�gina

\usepackage{lmodern}
\usepackage[T1]{fontenc}
\usepackage[spanish,activeacute]{babel}
\usepackage{mathtools}
\usepackage[spanish]{babel}
\usepackage[utf8]{inputenc}
\spanishdecimal{.}
\usepackage{color}

\begin{document}

Curso: \textit{ \textbf{Introducci\'on al Lenguaje de Programaci\'on R}}

\vspace{3mm}

{\color{red} {\Large Tarea 3}}. Entregar antes del 27 de febrero, 2018.

\vspace{3mm}

{\large $>>>>>>$ NOTA. Es parte de la tarea}

\vspace{3mm}

Profesor: L. Gonz\'alez-Santos\footnote{lgs@unam.mx}

\hrulefill

\begin{enumerate}
\item \textbf{Regresi\'on lineal simple}. Se desea encontra la ecuaci\'on de la recta que describe la serie de puntos $\{(X_i, Y_i) | i  = 1 ... n\}$, la ecuaci\'on esta dada por:

\[
{\displaystyle Y_{i}=\beta _{0}+\beta _{1}X_{i}+\varepsilon _{i}} {\displaystyle }
\]

donde ${\displaystyle \varepsilon _{i}} $ es el error asociado a la medici\'on del valor ${\displaystyle Y_{i}}$ y siguen los supuestos de modo que ${\displaystyle \varepsilon _{i}\sim N(0,\sigma ^{2})} $ (media cero, varianza constante e igual a un ${\displaystyle \sigma^2 } $ y ${\displaystyle \varepsilon _{i}\perp \varepsilon _{j}} $ con ${\displaystyle i\neq j} $).

\vspace{2mm}

Utilizando el m\'etodo de m\'inimos cuadrados se encuentra los valores de $\beta_0$ y $\beta_1$ mejor describen la recta: 

\vspace{2mm}

\[
{\displaystyle {\hat {\beta _{1}}}={\frac {\sum x\sum y-n\sum xy}{\left(\sum x\right)^{2}-n\sum x^{2}}}={\frac {\sum (x-{\bar {x}})(y-{\bar {y}})}{\sum (x-{\bar {x}})^{2}}}} 
\]

\[
{\displaystyle {\hat {\beta _{0}}}={\frac {\sum y-{\hat {\beta }}_{1}\sum x}{n}}={\bar {y}}-{\hat {\beta _{1}}}{\bar {x}}} 
\]

\begin{enumerate}
\item Aplicar la regresi\'on lineal simple a los puntos \{(1, 0), (2, 3), (3, 2), (4, 5), (5, 3), (6, 8), (7, 7), (8, 8), (9, 10), (10, 10), (11, 10), (12, 14)\}
\item Graficar los puntos y la recta ajustada.
\item A partir de la recta encontrar los valores en los puntos $x_1=6.5, x_2=0, x_3=17$
\end{enumerate}
\item Apartir de los 3 grupos de edades: 
G1= \{ 39, 59, 56, 47, 36, 48, 38, 53, 23, 49, 65, 69\}, G2=\{60, 67, 56, 65, 62, 58, 54, 73, 84, 78\}, 
G3=\{58, 27, 26, 57, 49, 49, 55, 62, 54, 36, 71, 50, 37, 66, 25\}, hacer una gr\'afica de barras sus medias, con sus respectivos errores est\'andard.

\item A partir  de la poblaci\'on de edades: 38, 36, 45, 54, 18, 26, 21, 50, 51, 53, 18, 33, 49, 25, 29, 36, 65, 49, 49, 29,
54, 58, 40, 31, 30, 35, 56, 64, 54, 54, 40, 44, 51, 21, 25, 30, 52, 64, 48, 59,
61, 60, 26, 36, 36, 24, 32, 26, 20, 31, 23, 63, 42, 28, 45, 61, 61, 49, 32, 43,
52, 58, 42, 59, 59, 56, 56, 55, 62, 46, 45, 24, 58, 38, 57, 35, 18, 43, 30, 35,
43, 55, 30, 21, 22, 20, 60, 36, 52, 34, 26, 25, 60, 41, 18, 29, 28, 45, 56, 32. 

\vspace{2mm}

Tomar 30 muestras de tama\~no 10 sin repetici\'on de la poblaci\'on, hacer lo siguiente.

\begin{enumerate}
\item Calcular la media de cada muestra y hacer una gr\'afica de dispersi\'on de las 30 medias.

\item A\~nadir en la gr\'afica, lineas que muestren la media de la poblaci\'on y la media de las medias muestrales. 


\end{enumerate}

\end{enumerate}


\end{document}

